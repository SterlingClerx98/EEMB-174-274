\PassOptionsToPackage{unicode=true}{hyperref} % options for packages loaded elsewhere
\PassOptionsToPackage{hyphens}{url}
%
\documentclass[]{article}
\usepackage{lmodern}
\usepackage{amssymb,amsmath}
\usepackage{ifxetex,ifluatex}
\usepackage{fixltx2e} % provides \textsubscript
\ifnum 0\ifxetex 1\fi\ifluatex 1\fi=0 % if pdftex
  \usepackage[T1]{fontenc}
  \usepackage[utf8]{inputenc}
  \usepackage{textcomp} % provides euro and other symbols
\else % if luatex or xelatex
  \usepackage{unicode-math}
  \defaultfontfeatures{Ligatures=TeX,Scale=MatchLowercase}
\fi
% use upquote if available, for straight quotes in verbatim environments
\IfFileExists{upquote.sty}{\usepackage{upquote}}{}
% use microtype if available
\IfFileExists{microtype.sty}{%
\usepackage[]{microtype}
\UseMicrotypeSet[protrusion]{basicmath} % disable protrusion for tt fonts
}{}
\IfFileExists{parskip.sty}{%
\usepackage{parskip}
}{% else
\setlength{\parindent}{0pt}
\setlength{\parskip}{6pt plus 2pt minus 1pt}
}
\usepackage{hyperref}
\hypersetup{
            pdftitle={Day 5: Intro to Linear Regression},
            pdfauthor={Stephen R. Proulx},
            pdfborder={0 0 0},
            breaklinks=true}
\urlstyle{same}  % don't use monospace font for urls
\usepackage[margin=1in]{geometry}
\usepackage{color}
\usepackage{fancyvrb}
\newcommand{\VerbBar}{|}
\newcommand{\VERB}{\Verb[commandchars=\\\{\}]}
\DefineVerbatimEnvironment{Highlighting}{Verbatim}{commandchars=\\\{\}}
% Add ',fontsize=\small' for more characters per line
\usepackage{framed}
\definecolor{shadecolor}{RGB}{248,248,248}
\newenvironment{Shaded}{\begin{snugshade}}{\end{snugshade}}
\newcommand{\AlertTok}[1]{\textcolor[rgb]{0.94,0.16,0.16}{#1}}
\newcommand{\AnnotationTok}[1]{\textcolor[rgb]{0.56,0.35,0.01}{\textbf{\textit{#1}}}}
\newcommand{\AttributeTok}[1]{\textcolor[rgb]{0.77,0.63,0.00}{#1}}
\newcommand{\BaseNTok}[1]{\textcolor[rgb]{0.00,0.00,0.81}{#1}}
\newcommand{\BuiltInTok}[1]{#1}
\newcommand{\CharTok}[1]{\textcolor[rgb]{0.31,0.60,0.02}{#1}}
\newcommand{\CommentTok}[1]{\textcolor[rgb]{0.56,0.35,0.01}{\textit{#1}}}
\newcommand{\CommentVarTok}[1]{\textcolor[rgb]{0.56,0.35,0.01}{\textbf{\textit{#1}}}}
\newcommand{\ConstantTok}[1]{\textcolor[rgb]{0.00,0.00,0.00}{#1}}
\newcommand{\ControlFlowTok}[1]{\textcolor[rgb]{0.13,0.29,0.53}{\textbf{#1}}}
\newcommand{\DataTypeTok}[1]{\textcolor[rgb]{0.13,0.29,0.53}{#1}}
\newcommand{\DecValTok}[1]{\textcolor[rgb]{0.00,0.00,0.81}{#1}}
\newcommand{\DocumentationTok}[1]{\textcolor[rgb]{0.56,0.35,0.01}{\textbf{\textit{#1}}}}
\newcommand{\ErrorTok}[1]{\textcolor[rgb]{0.64,0.00,0.00}{\textbf{#1}}}
\newcommand{\ExtensionTok}[1]{#1}
\newcommand{\FloatTok}[1]{\textcolor[rgb]{0.00,0.00,0.81}{#1}}
\newcommand{\FunctionTok}[1]{\textcolor[rgb]{0.00,0.00,0.00}{#1}}
\newcommand{\ImportTok}[1]{#1}
\newcommand{\InformationTok}[1]{\textcolor[rgb]{0.56,0.35,0.01}{\textbf{\textit{#1}}}}
\newcommand{\KeywordTok}[1]{\textcolor[rgb]{0.13,0.29,0.53}{\textbf{#1}}}
\newcommand{\NormalTok}[1]{#1}
\newcommand{\OperatorTok}[1]{\textcolor[rgb]{0.81,0.36,0.00}{\textbf{#1}}}
\newcommand{\OtherTok}[1]{\textcolor[rgb]{0.56,0.35,0.01}{#1}}
\newcommand{\PreprocessorTok}[1]{\textcolor[rgb]{0.56,0.35,0.01}{\textit{#1}}}
\newcommand{\RegionMarkerTok}[1]{#1}
\newcommand{\SpecialCharTok}[1]{\textcolor[rgb]{0.00,0.00,0.00}{#1}}
\newcommand{\SpecialStringTok}[1]{\textcolor[rgb]{0.31,0.60,0.02}{#1}}
\newcommand{\StringTok}[1]{\textcolor[rgb]{0.31,0.60,0.02}{#1}}
\newcommand{\VariableTok}[1]{\textcolor[rgb]{0.00,0.00,0.00}{#1}}
\newcommand{\VerbatimStringTok}[1]{\textcolor[rgb]{0.31,0.60,0.02}{#1}}
\newcommand{\WarningTok}[1]{\textcolor[rgb]{0.56,0.35,0.01}{\textbf{\textit{#1}}}}
\usepackage{graphicx,grffile}
\makeatletter
\def\maxwidth{\ifdim\Gin@nat@width>\linewidth\linewidth\else\Gin@nat@width\fi}
\def\maxheight{\ifdim\Gin@nat@height>\textheight\textheight\else\Gin@nat@height\fi}
\makeatother
% Scale images if necessary, so that they will not overflow the page
% margins by default, and it is still possible to overwrite the defaults
% using explicit options in \includegraphics[width, height, ...]{}
\setkeys{Gin}{width=\maxwidth,height=\maxheight,keepaspectratio}
\setlength{\emergencystretch}{3em}  % prevent overfull lines
\providecommand{\tightlist}{%
  \setlength{\itemsep}{0pt}\setlength{\parskip}{0pt}}
\setcounter{secnumdepth}{0}
% Redefines (sub)paragraphs to behave more like sections
\ifx\paragraph\undefined\else
\let\oldparagraph\paragraph
\renewcommand{\paragraph}[1]{\oldparagraph{#1}\mbox{}}
\fi
\ifx\subparagraph\undefined\else
\let\oldsubparagraph\subparagraph
\renewcommand{\subparagraph}[1]{\oldsubparagraph{#1}\mbox{}}
\fi

% set default figure placement to htbp
\makeatletter
\def\fps@figure{htbp}
\makeatother


\title{Day 5: Intro to Linear Regression}
\author{Stephen R. Proulx}
\date{1/10/2021}

\begin{document}
\maketitle

\hypertarget{todays-objectives}{%
\section{Today's objectives:}\label{todays-objectives}}

\begin{itemize}
\tightlist
\item
  Learn the notation for describing Bayesian statistical models\\
\item
  Writing likelihoods for multiple observations of data\\
\item
  Simulating from a prior
\item
  grid approximation with 2 parameters\\
\item
  Calculating on the log scale
\end{itemize}

\hypertarget{notation}{%
\subsection{Notation}\label{notation}}

\begin{enumerate}
\def\labelenumi{\arabic{enumi}.}
\tightlist
\item
  Start with your likelihood. Usually this is a single line, but in rare
  cases (like if there are multiple types of data), it could be more.
  You can tell a line is part of the likelihood if it has data and
  parameters in it.
\end{enumerate}

In RMarkdown we can use \(\LaTeX\) to typeset equations. A nice intro to
\(\LaTeX\) is here:
\url{https://www.overleaf.com/learn/latex/Learn_LaTeX_in_30_minutes} .

I'll recreate the description on pate 77. We start typesetting a
\(\LaTeX\) equation with the ``\$\$'' symbol. The symbol
``\textasciitilde{}'' can be generated with the
\texttt{\textbackslash{}sim} command: \[
y_i \sim \mathrm{Normal}(\mu_i, \sigma)
\] Math symbols in greek are generally produced with
\texttt{lettername}.

\begin{enumerate}
\def\labelenumi{\arabic{enumi}.}
\setcounter{enumi}{1}
\tightlist
\item
  Next we put in any ``transformations'', which are sometimes called
  ``link'' functions. You can tell that a line is one of the
  transformations because it only involves parameters (but including
  hyper-parameters), not data, and because it does not involve a
  probability density (or the symbol \(\sim\))
\end{enumerate}

\[
\mu_i  = \beta  x_i
\] 3. And then then all the priors. Each true parameter has a prior. How
do you know it is a ``true'' paremter? Because it has a prior. The
priors are all probability statements, so they have the symbol \(\sim\),
and they no not inovled the data.\\
\[
\beta \sim \mathrm{Normal}(0,10) \\
\sigma  \sim \mathrm{Exponential}(1)\\
x_i \sim \mathrm{Normal}(0,1)
\]

\hypertarget{height-data-mean-and-variance}{%
\subsection{Height data, mean and
variance}\label{height-data-mean-and-variance}}

Here we will go through the example in the book that fits human height
data using a normal likelihood function. Because normal distributions
have both a mean and standard deviation, this is a two parameter model,
so a grid approximation will really be a grid this time.

\begin{Shaded}
\begin{Highlighting}[]
\KeywordTok{data}\NormalTok{(}\StringTok{"Howell1"}\NormalTok{)}
\NormalTok{d<-Howell1}
\NormalTok{d2<-}\StringTok{ }\NormalTok{d}\OperatorTok\StringTok{ }\KeywordTok{filter}\NormalTok{(age}\OperatorTok{>=}\DecValTok{18}\NormalTok{)}
\KeywordTok{ggplot}\NormalTok{(}\DataTypeTok{data=}\NormalTok{d2, }\KeywordTok{aes}\NormalTok{(}\DataTypeTok{x=}\NormalTok{height)) }\OperatorTok{+}\StringTok{ }\KeywordTok{geom_histogram}\NormalTok{(}\DataTypeTok{binwidth =} \FloatTok{2.5}\NormalTok{)}
\end{Highlighting}
\end{Shaded}

\includegraphics{January-20,-Linear-Models_files/figure-latex/loadData-1.pdf}

The model described in the book which we will fit: \$\$ y\_i
\sim \mathrm{Normal}(\mu, \sigma)\textbackslash{}
\mu \sim \mathrm{Normal}(178,20)\textbackslash{}
\sigma \sim \mathrm{Uniform}(0,50)

\$\$

A first important question is, what does the likelihood function really
mean, and why is it a good choice for this model? When we write \[
y_i \sim \mathrm{Normal}(\mu, \sigma)\\
\] what is actually meant is: \[
Pr(\mathrm{data|parmeters}) = \prod PDF(\mathrm{Normal}(y_i , \mu , \sigma))
\] This means that to get the likelihood of a dataset that invovles
multiple observations (which we label \(\y_i\)), we are multiplying
together the likelihood of each individual datapoint. This is because we
are assuming that each height is independent of each other, and the
joint probability of independent events is the product of their
probabilities.

An additional important point is that we can do our work on the log
scale, which converts products into sums, and then convert back to the
natural scale. This is largely a computational trick done in the
software behind the scenes.

\hypertarget{prior-predictive-simulation-of-height-data}{%
\subsubsection{Prior predictive simulation of height
data}\label{prior-predictive-simulation-of-height-data}}

It can be very useful to first see what sort of data, in broad terms,
your priors will produce. If they are producing absurd values, you have
good prior knowledge to exclude those parameters.

Here we apply the prior by drawing values for \(\mu\) and \(\sigma\) and
then drawing a normally distributed height from that.

\begin{Shaded}
\begin{Highlighting}[]
\NormalTok{prior_sim <-}\StringTok{ }\KeywordTok{tibble}\NormalTok{( }\DataTypeTok{mu=}\KeywordTok{rnorm}\NormalTok{(}\FloatTok{1e4}\NormalTok{,}\DataTypeTok{mean=}\DecValTok{178}\NormalTok{,}\DataTypeTok{sd=}\DecValTok{20}\NormalTok{), }\DataTypeTok{sigma=}\KeywordTok{runif}\NormalTok{(}\FloatTok{1e4}\NormalTok{,}\DataTypeTok{min=}\DecValTok{0}\NormalTok{,}\DataTypeTok{max=}\DecValTok{50}\NormalTok{)) }\OperatorTok
\StringTok{  }\KeywordTok{mutate}\NormalTok{(}\DataTypeTok{y=}\KeywordTok{rnorm}\NormalTok{(}\KeywordTok{n}\NormalTok{(),}\DataTypeTok{mean=}\NormalTok{mu,}\DataTypeTok{sd=}\NormalTok{sigma)) }
\end{Highlighting}
\end{Shaded}

Let's visualize it. It will be an over-dispersed normal, because we have
variance in the parameters and the normal sampling variability.

\begin{Shaded}
\begin{Highlighting}[]
\KeywordTok{ggplot}\NormalTok{(}\DataTypeTok{data=}\NormalTok{prior_sim, }\KeywordTok{aes}\NormalTok{(}\DataTypeTok{x=}\NormalTok{y))}\OperatorTok{+}\StringTok{ }\KeywordTok{geom_density}\NormalTok{()}
\end{Highlighting}
\end{Shaded}

\includegraphics{January-20,-Linear-Models_files/figure-latex/unnamed-chunk-2-1.pdf}

\hypertarget{grid-approximation-of-the-posterior}{%
\subsubsection{Grid approximation of the
posterior}\label{grid-approximation-of-the-posterior}}

Now we can do a grid approximation to generate the posterior, and in
this case we actually have two parameters so we atually have a grid.

In terms of coding, the trick here is to use \texttt{expand} to produce
all combinations of two columns.

\begin{Shaded}
\begin{Highlighting}[]
\CommentTok{#code to grid out the posterior}

\NormalTok{n <-}\StringTok{ }\DecValTok{200} \CommentTok{# how many steps to use in the grid.  }

\NormalTok{d_grid <-}
\StringTok{  }\KeywordTok{tibble}\NormalTok{(}\DataTypeTok{mu    =} \KeywordTok{seq}\NormalTok{(}\DataTypeTok{from =} \DecValTok{150}\NormalTok{, }\DataTypeTok{to =} \DecValTok{160}\NormalTok{, }\DataTypeTok{length.out =}\NormalTok{ n),}
         \DataTypeTok{sigma =} \KeywordTok{seq}\NormalTok{(}\DataTypeTok{from =} \DecValTok{4}\NormalTok{,   }\DataTypeTok{to =} \DecValTok{9}\NormalTok{,   }\DataTypeTok{length.out =}\NormalTok{ n)) }\OperatorTok\StringTok{ }
\StringTok{  }\CommentTok{# expand can be used to combine all the elements from two rows}
\StringTok{  }\KeywordTok{expand}\NormalTok{(mu, sigma)}
\end{Highlighting}
\end{Shaded}

Have a quick look at the grid to see how it worked.

\begin{Shaded}
\begin{Highlighting}[]
\KeywordTok{view}\NormalTok{(d_grid)}
\end{Highlighting}
\end{Shaded}

We need to write a special function to calculate our likelihood. This
function takes as input the values of \(\mu\) and \(\sigma\) that we are
considereing. It also needs to use the data, in our case still stored in
the dataframe \texttt{d2}.

We code this by summing up the log likelihoods

\begin{Shaded}
\begin{Highlighting}[]
\NormalTok{height_lik_f <-}\StringTok{ }\ControlFlowTok{function}\NormalTok{(mu_input,sigma_input)\{}
  \KeywordTok{sum}\NormalTok{(}\KeywordTok{dnorm}\NormalTok{(}
\NormalTok{  d2}\OperatorTok{$}\NormalTok{height , }
  \DataTypeTok{mean=}\NormalTok{mu_input,}
  \DataTypeTok{sd=}\NormalTok{sigma_input,}
  \DataTypeTok{log=}\OtherTok{TRUE}\NormalTok{ ))}
\NormalTok{\}}
\end{Highlighting}
\end{Shaded}

And we convert this to a ``vectorized'' function so we can use it in
\texttt{dplyr} functions.

\begin{Shaded}
\begin{Highlighting}[]
\NormalTok{height_lik_f_vec <-}\StringTok{ }\KeywordTok{Vectorize}\NormalTok{(height_lik_f)}
\end{Highlighting}
\end{Shaded}

Repeat your mantra: likelihood * prior and then normalize! This time we
do it on the log scale and then convert back to the natural scale.

\begin{Shaded}
\begin{Highlighting}[]
\NormalTok{posterior_table <-}\StringTok{ }\NormalTok{d_grid }\OperatorTok
\StringTok{  }\KeywordTok{mutate}\NormalTok{(}\DataTypeTok{log_likelihood=}\KeywordTok{height_lik_f_vec}\NormalTok{(mu,sigma),}
         \DataTypeTok{log_prior_mu =} \KeywordTok{dnorm}\NormalTok{(mu,    }\DataTypeTok{mean =} \DecValTok{178}\NormalTok{, }\DataTypeTok{sd  =} \DecValTok{20}\NormalTok{, }\DataTypeTok{log =}\NormalTok{ T),}
         \DataTypeTok{log_prior_sigma =} \KeywordTok{dunif}\NormalTok{(sigma, }\DataTypeTok{min  =} \DecValTok{0}\NormalTok{,   }\DataTypeTok{max =} \DecValTok{50}\NormalTok{, }\DataTypeTok{log =}\NormalTok{ T),}
         \DataTypeTok{raw_log_posterior =}\NormalTok{ log_likelihood }\OperatorTok{+}\StringTok{ }\NormalTok{log_prior_mu }\OperatorTok{+}\StringTok{ }\NormalTok{log_prior_sigma,}
         \DataTypeTok{log_posterior =}\NormalTok{ raw_log_posterior }\OperatorTok{-}\StringTok{ }\KeywordTok{max}\NormalTok{(raw_log_posterior) ,}
         \DataTypeTok{raw_posterior =} \KeywordTok{exp}\NormalTok{(log_posterior),}
         \DataTypeTok{posterior =}\NormalTok{ raw_posterior}\OperatorTok{/}\KeywordTok{sum}\NormalTok{(raw_posterior))  }
\end{Highlighting}
\end{Shaded}

\hypertarget{exploring-the-posterior-probability}{%
\subsubsection{Exploring the posterior
probability}\label{exploring-the-posterior-probability}}

We can view the posterior probability, which has parameters in two
dimensions, using a contour plot. This figure uses the calculated
probabilities, not samples from the posterior distribuiton.

\begin{Shaded}
\begin{Highlighting}[]
\KeywordTok{contour_xyz}\NormalTok{(posterior_table}\OperatorTok{$}\NormalTok{mu, posterior_table}\OperatorTok{$}\NormalTok{sigma , posterior_table}\OperatorTok{$}\NormalTok{posterior)  }
\end{Highlighting}
\end{Shaded}

\includegraphics{January-20,-Linear-Models_files/figure-latex/view_posterior-1.pdf}

We sample from the posterior in exactly the same way as before. Each row
of our dataframe contains values for both \(\mu\) and \(\sigma\).

\begin{Shaded}
\begin{Highlighting}[]
\NormalTok{samples_height_model <-}\StringTok{ }\KeywordTok{sample_n}\NormalTok{(posterior_table, }\DataTypeTok{weight =}\NormalTok{posterior, }\DataTypeTok{size=}\FloatTok{1e4}\NormalTok{, }\DataTypeTok{replace=}\OtherTok{TRUE}\NormalTok{) }\OperatorTok
\StringTok{  }\KeywordTok{select}\NormalTok{(mu,sigma)}
\end{Highlighting}
\end{Shaded}

We can view a summary with the \texttt{precis} command. This gives a
table with means and quantiles, and also a chunky little histogram.

\begin{Shaded}
\begin{Highlighting}[]
\KeywordTok{precis}\NormalTok{(samples_height_model)}
\end{Highlighting}
\end{Shaded}

\begin{verbatim}
##             mean        sd       5.5%      94.5%   histogram
## mu    154.604508 0.4137554 153.969849 155.276382    ▁▁▁▅▇▂▁▁
## sigma   7.768226 0.2971587   7.316583   8.271357 ▁▁▂▅▇▇▃▂▁▁▁
\end{verbatim}

Now that we have samples, we can view them in a number of ways.

We can look at the scatter plot of the points themselves. This is a fine
thing to glance at.

\begin{Shaded}
\begin{Highlighting}[]
\KeywordTok{ggplot}\NormalTok{(}\DataTypeTok{data=}\NormalTok{samples_height_model, }\KeywordTok{aes}\NormalTok{(}\DataTypeTok{x=}\NormalTok{mu,}\DataTypeTok{y=}\NormalTok{sigma)) }\OperatorTok{+}\StringTok{ }
\StringTok{  }\KeywordTok{geom_point}\NormalTok{(}\DataTypeTok{size =} \FloatTok{.9}\NormalTok{, }\DataTypeTok{alpha =} \FloatTok{0.1}\NormalTok{) }\OperatorTok{+}
\StringTok{  }\KeywordTok{scale_x_continuous}\NormalTok{(}\DataTypeTok{limits =} \KeywordTok{c}\NormalTok{(}\DecValTok{153}\NormalTok{,}\FloatTok{156.5}\NormalTok{),}\DataTypeTok{breaks=}\KeywordTok{seq}\NormalTok{(}\DataTypeTok{from=}\DecValTok{153}\NormalTok{,}\DataTypeTok{to=}\DecValTok{156}\NormalTok{,}\DataTypeTok{by=}\DecValTok{1}\NormalTok{),}\DataTypeTok{labels=}\KeywordTok{c}\NormalTok{(}\StringTok{"153.0"}\NormalTok{,}\StringTok{"154.0"}\NormalTok{,}\StringTok{"155.0"}\NormalTok{,}\StringTok{"156.0"}\NormalTok{))}\OperatorTok{+}
\StringTok{  }\KeywordTok{scale_y_continuous}\NormalTok{(}\DataTypeTok{limits =} \KeywordTok{c}\NormalTok{(}\FloatTok{6.5}\NormalTok{,}\FloatTok{9.0}\NormalTok{), }\DataTypeTok{breaks=}\KeywordTok{seq}\NormalTok{(}\DataTypeTok{from =}\DecValTok{7}\NormalTok{,}\DataTypeTok{to=}\DecValTok{9}\NormalTok{,}\DataTypeTok{by=}\FloatTok{0.5}\NormalTok{))}
\end{Highlighting}
\end{Shaded}

\begin{verbatim}
## Warning: Removed 1 rows containing missing values (geom_point).
\end{verbatim}

\includegraphics{January-20,-Linear-Models_files/figure-latex/unnamed-chunk-6-1.pdf}

We can view a contour plot of the samples:

\begin{Shaded}
\begin{Highlighting}[]
\KeywordTok{ggplot}\NormalTok{(}\DataTypeTok{data=}\NormalTok{samples_height_model, }\KeywordTok{aes}\NormalTok{(mu, sigma)) }\OperatorTok{+}
\StringTok{  }\KeywordTok{geom_density_2d}\NormalTok{(}\DataTypeTok{bins=}\DecValTok{10}\NormalTok{) }
\end{Highlighting}
\end{Shaded}

\includegraphics{January-20,-Linear-Models_files/figure-latex/unnamed-chunk-7-1.pdf}

\hypertarget{comparing-the-marginal-plots}{%
\subsubsection{Comparing the marginal
plots}\label{comparing-the-marginal-plots}}

What do the ``marginal'' densities mean? They tell us how one parameter
is distributed if we don't know the value of the other parameters.

From samples, we can get them from our samples without any real work by
just focusing on one parameter at a time:

\begin{Shaded}
\begin{Highlighting}[]
\KeywordTok{ggplot}\NormalTok{(}\DataTypeTok{data=}\NormalTok{samples_height_model, }\KeywordTok{aes}\NormalTok{(}\DataTypeTok{x=}\NormalTok{mu)) }\OperatorTok{+}\StringTok{ }
\StringTok{  }\KeywordTok{geom_density}\NormalTok{() }
\end{Highlighting}
\end{Shaded}

\includegraphics{January-20,-Linear-Models_files/figure-latex/unnamed-chunk-8-1.pdf}

\begin{Shaded}
\begin{Highlighting}[]
\KeywordTok{ggplot}\NormalTok{(}\DataTypeTok{data=}\NormalTok{samples_height_model, }\KeywordTok{aes}\NormalTok{(}\DataTypeTok{x=}\NormalTok{sigma)) }\OperatorTok{+}\StringTok{ }
\StringTok{  }\KeywordTok{geom_density}\NormalTok{() }
\end{Highlighting}
\end{Shaded}

\includegraphics{January-20,-Linear-Models_files/figure-latex/unnamed-chunk-8-2.pdf}

Lots of packages have methods for plotting samples from posteriors. One
common method is some kind of ``pair'' plot, which combines a scatter
plot of each pair of parameters and a marginal density plot of each
individual parameter. We'll use the \texttt{bayesplot} package version,
with some specific options that make it look nicer.

\begin{Shaded}
\begin{Highlighting}[]
\NormalTok{bayesplot}\OperatorTok{::}\KeywordTok{mcmc_pairs}\NormalTok{(samples_height_model,}\DataTypeTok{diag_fun =} \StringTok{"dens"}\NormalTok{,}
  \DataTypeTok{off_diag_fun =}  \StringTok{"hex"}\NormalTok{) }
\end{Highlighting}
\end{Shaded}

\begin{verbatim}
## Warning: Only one chain in 'x'. This plot is more useful with multiple
## chains.
\end{verbatim}

\includegraphics{January-20,-Linear-Models_files/figure-latex/unnamed-chunk-9-1.pdf}

\hypertarget{exercise-quantify-the-distributions}{%
\subsubsection{Exercise: Quantify the
distributions}\label{exercise-quantify-the-distributions}}

All of the methods we've used for quantifying a single paremeter's
posterior distribution can still be used in the same way as before. For
both \(\mu\) and \(\sigma\), calculate the mean, median, 5 and 95\%
quantiles, and the HPDIs.

\end{document}
